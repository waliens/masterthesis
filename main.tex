\documentclass[a4paper,12pt]{report}
\usepackage[utf8]{inputenc}
\usepackage[english]{babel}
\usepackage{amssymb}
\usepackage{amsmath}
\usepackage{amsthm}
\usepackage{mathrsfs}
\usepackage[nottoc]{tocbibind}
\usepackage[backend=biber,style=alphabetic]{biblatex}
\usepackage{graphicx}
\usepackage[left=3cm, right=3cm, top=3cm, bottom=3cm]{geometry}
\usepackage{listings}
\usepackage{algpseudocode}
\usepackage[bookmarks = false,hidelinks]{hyperref}
\usepackage{color}
\usepackage{subfigure}
\usepackage{multirow}
\usepackage{enumitem}

\newtheorem{algorithm}{Algorithm}
\newtheorem{definition}{Definition}

\setcounter{secnumdepth}{3} % include subsubsection in section numbering
\setcounter{tocdepth}{3} % include subsubsection in toc

\setitemize{itemsep=1pt} % reduce vertical spacing

%% Create python environment
%% http://tex.stackexchange.com/questions/199375/problem-with-listings-package-for-python-syntax-coloring

% Defining colors
\definecolor{deepblue}{rgb}{0,0,0.5}
\definecolor{deepred}{rgb}{0.6,0,0}
\definecolor{deepgreen}{rgb}{0,0.5,0}

\newcommand\pythonstyle{\lstset{
  language=Python,
  backgroundcolor=\color{white}, %%%%%%%
  basicstyle=\footnotesize\ttfamily,
  otherkeywords={self,def,for,@property},            
  keywordstyle=\bf\color{deepblue},
  emph={Segmenter,ColorClassifier,PolygonClassifier,SquareRule,CircleRule,DispatchingRule,CustomSegmenter,WorkflowBuilder,NumpyImage,__init__},          
  emphstyle=\bf\color{deepred},    
  stringstyle=\color{deepgreen},
  commentstyle=\color{red},  %%%%%%%%            
  showstringspaces=false            
}}

% Python environment
\lstnewenvironment{python}[1][]
{
\pythonstyle
\lstset{#1}
}
{}

%% To uncomment if needed
%\usepackage{array}
%\usepackage[usenames,dvipsnames]{color}
%\usepackage{arydshln}
%\usepackage{slashbox}
%\usepackage{pdflscape}
%\usepackage{cancel}
%\setlist[itemize,1]{label=$\bullet$}

%% Add bibliography database
\addbibresource{bibliography.bib}

%% Set document data
\author{Mormont Romain}
\title{A workflow for computer-aided cytology and its applications}
\date{Academic year 2015-2016}

\begin{document}
	
	% Start frontpage =========================
	\thispagestyle{empty}
{ \sf

\begin{center}
	{\small University of Liège}\\
	{\small Faculty of Applied Sciences}
\end{center}

\vfill

% Logo 
\begin{figure}[!h]
	\center
	\includegraphics[scale=0.2]{image/institution_ulg.png}
\end{figure}

\vfill

\begin{center}
	{\LARGE Master Thesis\\}
\end{center}

\noindent\rule{1\linewidth}{1px}

\begin{center}
	{\LARGE \sf \textbf{A workflow for large-scale computer-aided cytology and its applications}.}
\end{center}

\noindent\rule{1\linewidth}{1px}

\begin{center}
	\textit{Author} : Romain Mormont
\end{center}

\begin{center}
	\textit{Supervisor} : Dr. Raphaël Marée
\end{center}

\begin{center}
	\textit{Academic} : Prof. Pierre Geurts
\end{center}
\vfill

\begin{center}
	Master thesis submitted for the degree of \\
	\vspace{0.25cm}
	{\Large MSc in Computer Science and Engineering}
\end{center}

\vfill
\begin{center}
Academic year 2015-2016\\
\end{center}
}
	% End frontpage ===========================
	
	\newpage
	
	% Start table of content ==================
	\clearpage
	\setcounter{page}{1}
	\tableofcontents
	% End table of content ====================
	
	\newpage

	% Start summary ===========================
	\chapter*{Summary}
	\begin{center}
 
	{\sf \large \textbf{A workflow for large-scale computer-aided cytology and its applications}.}\\
\raggedleft
 	{\sf \textit{by Romain Mormont} }\\
\centering
	{\sf \textit{Supervisor}: Dr. Raphaël Marée, \textit{Academic}: Prof. Pierre Geurts }\\
\centering
	{\sf Academic year 2015-2016 }\

\end{center}
\hfill

In several fields of application, multi-gigapixel images must be analysed to gather information and take decision. This analysis is often performed manually, which is a tedious task given the volume of data to process. For instance, in cytology, branch of medical sciences which focuses on study of cells, cytopathologists analyse cell samples microscope slides in order to diagnose diseases such as cancers. Typically, malignancy is assessed by the presence or absence of cells with given characteristics. In geology, climate variations can be analysed by studying the concentration 
of micro-organisms in core samples. The concentration is usually evaluated by smearing the samples onto microscope glass slides and counting those micro-organisms. \\

In those situations, computer sciences and, especially, machine learning and image processing provide a great alternative to a pure-human approach as they can be used to extract relevant information automatically. Especially, those kinds\\
 of problems can be expressed as object detection and classification problems.
\\

This thesis presents the elaboration and assessment of a generic framework, \textit{SLDC}, for object detection and classification in multi-gigapixel images. Especially, this framework provides implementers with a concise way of formulating problem dependent-components of their algorithm (i.e. segmentation and classification) while it takes care of problem-independent concerns such as parallelization and large image handling. 
\\

The performances of the framework are then assessed on a real-world problem, thyroid nodule malignancy. Especially, a workflow is built to detect malignant cells in thyroid cell samples whole-slides.
\\

Results are promising: the effective processing time for an image containing 8 gigapixels is less than 10 minutes. In order, to further reduce this execution time, some improvements are proposed.
\\

The framework implementation can be found on GitHub: \url{https://github.com/waliens/sldc}.

\hfill
	\newpage
	% End summary =============================
	
	% Start acknowledgement ===================
	\chapter*{Acknowledgement}
	First, I would like to thank Raphaël Marée for the time he dedicated to guide and advise me throughout the making of this thesis. His support has been a great help and has undoubtedly contributed to the successful completion of this work. 
\\

Then, I would like to thank Pierre Geurts for his guidance and, especially, for his advice regarding the redaction of this manuscript. I am also grateful to him for introducing me to Raphaël Marée and therefore allowing me to work on such an interesting subject. 
\\

I would also like to thank Renaud Hoyoux for his availability and quickness at solving bugs which sometimes prevented be from getting further.
\\

A great thank to Jean-Michel Begon for his availability to answer my question about his implementation. 
\\

Finally, I would like to address my gratitude to my friends and family for their support. Especially, I address a special thanks to Floriane, Justine and Fabrice who have accepted to proofread this manuscript.



	\newpage
	% End acknowledgement =====================
	
	% Start introduction ======================
	\chapter{Introduction}
	In several domains, multi-gigapixel images must be analysed for the purpose of gathering information and/or for taking decisions. Typically, the information is represented by the presence of a series of objects of interest which are embedded into the image. The aim of the analysis is to locate and identify those objects. Depending on the problem and  specific field of application, the extracted objects can be used for various purposes. For instance, in cytology, digitized microscope slides containing human tissues are analysed by physicians in order to diagnose particular diseases, the disease in question manifesting itself by the presence of cells having certain characteristics. In geology, slides containing core samples can be digitized and analysed to find concentration of certain micro-organisms.

Those images are usually analysed manually by experts. However, due to the size of the problem, the analysis is not always performed exhaustively. When possible, experts typically select a reduced number of regions to study and draw conclusions from the observations performed in those regions. This process has obviously the drawback of increasing the risk of missing objects of interest.

Because of the risk yielded by the previous method and because manual analysis of full images is long and tedious, computer programs could be used to assist experts. For instance, those programs could indicate which regions are worth analysing and which are not. They could also perform the search for the expert under his supervision: that is, the expert would be able to provide a feedback to the program which could then improve its detection process. 

In order to provide this assistance and to learn from experts' feedbacks, image processing and machine learning are used. Whereas IP and ML provide a complete toolbox of algorithms for computer vision in general, they are however not necessarily well suited for handling large images. Especially, typical implementations of those algorithms implicitly make the assumption that the full image can be loaded into memory which is not always possible. Indeed, multi-gigapixel images typically require several gigabytes of memory. The execution times of those algorithms generally grow with the size of the image, yielding unacceptable execution times. Parallelism can alleviate this problem but, again, typical implementations do not necessarily support this feature. Therefore, when diving into a new problem of object detection, implementers typically develop workflows by combining machine learning and image processing algorithms to handle detection but they also have to deal with problem-independent concerns such as parallelism or memory constraints. 

This thesis proposes \textit{SLDC}, a generic framework for solving problems of object detection and classification in multi-gigapixel images. Especially, it provides implementers with a structure to define problem-dependent components of the algorithm (i.e. detection and classification) in a concise way. Every other concerns such as parallelization and large image handling are encapsulated by the framework. It also provides a way to execute several processing workflows one after another on the same image as well as a powerful and customizable logging system. Typically, when facing a new problem of object detection and classification, an implementer instantiates the framework into a workflow to deal with this problem. 

In Chapter \ref{chap:context}, the problem of object detection and classification is introduced and its application to different cases is presented. In Chapter \ref{chap:work_intro}, the framework and its implementation are presented. In Chapter \ref{chap:thyroid}, the framework is applied to a cytology problem, the thyroid case.
	\newpage
	% End introduction ========================
	
	% Start chapter 1 : obj. detection ========
	\chapter{Object detection in large images}
	\section{General problem}
\subsection{Formulation}
Generic formulation of the object detection problem
\subsection{Implementation issues}
What issues an implementor could face when trying to implement object detection in large images
\subsection{Related works}
What solutions are usually presented in the litterature to solve those problems (shallow overview as this is a wide topic) 
\section{Cytology}
\subsection{An object detection problem ?}
Why it is an instance of the "object detection in large images" problem
\subsection{The thyroid case}
Explanation of the Thyroid case 
	\newpage
	% End chapter 1 : obj. detection ==========	
	
	% Start chapter 2 : generic workflow ======
	\chapter{A generic workflow : Segment Locate Dispatch Classify}
	In this chapter, a generic workflow for solving problems of objects detection and classification in images is presented. The history of this workflow is presented in Section \ref{sec:history_first_dev}. Section \ref{sec:workflow_principle} introduces the workflow itself. Finally, the implementation carried out in the context of this thesis is presented and discussed in Section \ref{sec:workflow_impl}.

\section{History and first developments}
\label{sec:history_first_dev}
The Segment-Locate-Dispatch-Classify (SLDC) workflow was first imagined by ?? Jean-Michel Begon ?? as a generalization of the work on thyroid nodule malignancy detection made in \cite{adeblire2013}. In the context of his master thesis, the author had implemented a processing workflow for detecting cells with inclusion and proliferative architectural patterns (see ?? (thyroid)) in digitized thyroid punctions slides. The cells and architectural patterns were detected by segmenting the images and then classified using machine learning techniques. As explained in the Section ?? (thyroid), some patterns could themselves contain cells with inclusion. Therefore, the author implemented a second processing workflow to detect those cells. This workflow was similar to the first because it relied on a segmentation algorithm to isolate cells in patterns and then used machine learning to assess their malignity. 

From those workflows, a common pattern emerged: performing detection using a segmentation algorithm and then classifying the detected objects using machine learning techniques. In 2015, ?? Jean-Michel Begon ?? developed a first version of a generic workflow based on this pattern and gave it the name \textit{Segment-Locate-Dispatch-Classify}. Then, he applied its workflow to the thyroid case. Unfortunately, this implementation suffered from some drawbacks which made it hard to reuse in other contexts. The workflow was therefore re-worked in the context of this master thesis.

\section{Principle}
\label{sec:workflow_principle}

\subsection{Algorithm}
\label{ssec:workflow_algo}
The workflow is a meta-algorithm\footnote{In this context, a meta-algorithm is an algorithm that coordinates the execution of other algorithms.} that detects and classifies objects contained in images. Particularly, given a two-dimensional\footnote{A third dimension can be dedicated to the images channel (i.e. 3 channels for RGB images, 4 channels for RGBA images).} image $\mathcal{I}$ as input, it is expected to output the information about the objects of interest in this image. Those information include the shape of the object, its location in the image as well as a classification label. Formally, the workflow can be seen as an operator $\mathcal{W}$:

\begin{definition} Let $\mathcal{W}$ be an operator such that 
	\begin{equation}\label{eqn:workflow_operator}
		\mathcal{W}(\cdot) : \mathcal{I} \rightarrow \left\{(o_1,C_1),...,(o_N, C_N)\right\}
	\end{equation}
	where $N$ is the number of objects of interest in $\mathcal{I}$ and $(o_i, C_i)$ is a tuple. The first element of this tuple, $o_i$, is a representation of the information (shape and location) about the $i^{th}$ object of interest found in $\mathcal{I}$ and the second, $C_i$, its classification label. 
\end{definition}

It is worth noting that genericity is of the essence. That is, the meta-algorithm should be able to solve the widest possible range of object detection and classification problems. Moreover, as explained in Section \ref{sec:history_first_dev}, it should produce those outputs using image segmentation and machine learning. As far as the segmentation is concerned, genericity is usually hard to obtain because of the high variability of images across different problems. In order to ensure genericity, the workflow doesn't impose a particular segmentation procedure but expects the implementer to provide one that suits the problem. The same goes for the classification models used for predicting the labels of the objects. 

In the subsequent sections, some additional operators are defined and used to build the $\mathcal{W}$ operator. First, a basic version of the algorithm is presented and then refined in order to achieve an acceptable level of genericity.

\subsection{Additional operators}
\label{ssec:other_operators}

Segmentation is the first operation applied to the image. This step of the algorithm is where the detection is actually carried out:
 
\begin{definition} \label{def:segmentation_op}
Let $\mathcal{S}$ be the \textbf{segment} operator. It is applied to an image $\mathcal{I}$ and produces a binary mask $\mathcal{B}$. The pixel $b_{ij}$ of $\mathcal{B}$ is 1 if the pixel $p_{ij}$ of $\mathcal{I}$ is located in an object of interest, otherwise it is 0. Formally:
\begin{equation}
	\label{eqn:operator_segment}
	\mathcal{S}(\cdot) : \mathcal{I} \rightarrow \mathcal{B}
\end{equation}
\end{definition}

While the segmented image theoretically contains the necessary information about the detected objects (i.e. shape and position in the image), the format of this information is inconvenient to query mostly because it is embedded into the binary mask and a single object cannot be trivially extracted. An intermediate step that would convert this information into a more convenient format is therefore needed. This format should encode both the shape of the object and its position in the image. It appears that polygons match this specification. 

\begin{definition} \label{def:locate_op}
Let $\mathcal{L}$ be the \textbf{location} operator. It is applied to a binary mask and produces a set of polygons encoding the shapes and positions of every object in the image. Formally:

\begin{equation}
	\mathcal{L}(\cdot) : \mathcal{B} \rightarrow \left\{P_1, ..., P_N\right\}
\end{equation}

where $\mathcal{B}$ is a binary mask as defined in Definition \ref{def:segmentation_op}, $N$ is the number of objects of interest in $\mathcal{B}$ and $P_i$ is the polygon representing the geometric contour of the $i^{th}$ object in $\mathcal{B}$.
\end{definition}

The final step of the workflow is the object classification and is performed by a classifier which is passed a representation of the object (e.g. image, geometric information,...) and produces a classification label. In this theory, there is no restriction about the nature or representation of the objects processed by the classifiers.

\begin{definition} Let $\mathcal{T}$ be the \textbf{classifier} operator. It is applied to an object of interest and produces a classification label. Formally:
\begin{equation}
	\mathcal{T}(\cdot) : o \rightarrow C
\end{equation}
where $o$ is the object and $C$, the classification label. 
\end{definition}
\begin{definition}
Let $\mathcal{T}^*$ be an extension of $\mathcal{T}$ which is given a set of objects and produces labels for all of them. Formally: 
\begin{equation}
	\mathcal{T}^*(\cdot) : \left\{o_1, ..., o_N\right\}  \rightarrow \left\{C_1, ..., C_N\right\}
\end{equation}
\end{definition}

\subsection{Single segmentation, single classifier}
\label{ssec:single_single}

The most simple construction of $\mathcal{W}$ would be the composition of the operators defined in Section \ref{ssec:other_operators}. Particularly, the compositions $\mathcal{S} \circ \mathcal{L}$ and $\mathcal{S} \circ \mathcal{L} \circ \mathcal{T}^*$ would respectively produce the polygons representing the objects and and their labels. This construction is summarized in Algorithm \ref{algo:single_seg_single_classif}: 

\begin{algorithm} \label{algo:single_seg_single_classif} 
	Construction of $\mathcal{W}$ using one segmentation and one classifier:
	
	\begin{enumerate}
		\item Return $\left(\mathcal{S} \circ \mathcal{L}\right)\left(\mathcal{I}\right) \times \left(\mathcal{S} \circ \mathcal{L} \circ \mathcal{T}^*\right)\left(\mathcal{I}\right)$
	\end{enumerate}
\end{algorithm}

As explained in Section \ref{ssec:workflow_algo}, the definition of $\mathcal{S}$ and $\mathcal{T}^*$ would be left at the implementer's hands. As far as the $\mathcal{L}$ operator is concerned, it could be imposed by the workflow without loss of genericity. Such an construction of $\mathcal{W}$ could already solve any object detection and classification problem on image in which the labels can be predicted by a single classifier. However, in some cases, one classifier is not enough. This happen, for instance, when the image contains objects of very different nature and using several classifiers would yield better results than using a single one. An extension is therefore needed.

\subsection{Single segmentation, several classifiers}
\label{ssec:single_several}

In this attempt to construct a generic $\mathcal{W}$ operator, the image is assumed to contain $M$ distinct types of objects and the workflow uses $M$ classifiers (the $i^{th}$ classifier being noted as $\mathcal{T}_i$ with $i \in \{1,...,M\}$) to classify those objects. As an object should only be processed by one classifier, the workflow has to be added a new step which consists in dispatching each polygon to its most appropriate classifier. 

\begin{definition}\label{def:dispatch_op} 
	Let $\mathcal{D}$ be the dispatch operator. It is applied to a polygon and produces an integer which identifies the most appropriate classifier for processing this polygon: 

	\begin{equation}
		\mathcal{D}(\cdot) : P \rightarrow i, i \in \{1,...,M\}
	\end{equation}
\end{definition}

This step being problem dependent, it is the responsibility of the implementer to define the rules used for dispatching the polygons. However, the format of these rules can be defined.

\begin{definition} 
	Let $\mathcal{P}$ be a set of $M$ predicates $p_1, ..., p_M$ which associate truth values to polygons:
	\begin{equation}
		p_i(\cdot) : P \rightarrow t \in \{true, false\}, i \in \left\{1,...,M\right\} 
	\end{equation}
	where $p_i$ is the predicate associated with the $i^{th}$ classifier. The polygon $P$ is dispatched to a classifier $\mathcal{T}_i$ if $p_i$ associates true to this polygon. To avoid dispatching an object to several classifier, the predicates should verify the following property:
	\begin{equation}
		p_i = true \Leftrightarrow p_j = false, \forall j \neq i
	\end{equation} 
\end{definition}

Given this format, the $\mathcal{D}$ operator can be trivially constructed as it returns $i$ if $p_i$ is \textit{true}. The algorithm resulting from this construction of $\mathcal{W}$ starts the same way as in Section \ref{ssec:single_single}: the image is applied the segment and locate operators. Then, the resulting polygons are dispatched and classified to produce the classification label. The resulting algorithm is summarized in Algorithm \ref{algo:single_seg_several_classif}. 

While the range of problems that can be solved using this algorithm has been increased compared to the version with a single classifier, there are still some problems that cannot be. In particular, if some objects are included into other bigger objects, they won't be considered as independent objects. 

Before extending the algorithm for handling this case, it is worth noting that Algorithm \ref{algo:single_seg_several_classif} is completely compatible with Algorithm \ref{algo:single_seg_single_classif}. Indeed, if there is only one classifier (i.e. $M = 1$) and the predicate $p_1$ always returns $true$, then both algorithms are exactly the same. 

\begin{algorithm}\label{algo:single_seg_several_classif}
Construction of the $\mathcal{W}$ operator with a single segmentation and several classifiers. 
\begin{enumerate}
	\item Apply the $\mathcal{S} \circ \mathcal{L}$ composition to the input image $\mathcal{I}$ to extract the objects of interest as the set of polygons $S_p \leftarrow \left\{P_1, ..., P_N \right\}$
	\item Initialize the labels set $L \leftarrow \emptyset$
	\item For each polygon $P \in S_p$:
	\begin{enumerate}
		\item Compute the classification label $C \leftarrow \mathcal{T}_{\mathcal{D}(P)}(P)$
		\item Place the label in the labels set $L \leftarrow L \cup \{C\}$
	\end{enumerate}
	\item Build and return objects and labels set $S_p \times L$.
\end{enumerate}
\end{algorithm}

\subsection{Chaining workflows}

To handle the case when objects of interest can be included into objects of bigger size in the image, a solution consists in executing several instances of Algorithm \ref{algo:single_seg_several_classif} sequentially one after another. 

\begin{definition}\label{def:several_w_op}
	Let $\mathcal{W}_1, ..., \mathcal{W}_K$ be a set of $K$ instances of Algorithm \ref{algo:single_seg_several_classif}. Each algorithm $\mathcal{W}_i$ has its own segmentation procedure $\mathcal{S}_i$ and proper sets of dispatching predicates $\mathcal{P}_i$ and classifiers $S_{\mathcal{T},i}$.
\end{definition}

While $\mathcal{W}_1$ would be applied to the full image $\mathcal{I}$ to extract all the objects of interest, $\mathcal{W}_2, ..., \mathcal{W}_K$ are only passed image windows containing the previously detected objects.

\begin{definition}\label{def:image_window}
	Let $\mathcal{I}_P$ be an image window extracted from image $\mathcal{I}$ and containing the object represented by the polygon $P$. The window is the minimum bounding box containing this polygon.
\end{definition}

A further refinement would be to provide a way for the implementer to filter the polygons of which the windows are passed to a given workflow instance. Indeed, a given instance $\mathcal{W}_i$ might be designed to process only a certain category of objects and therefore should not be passed windows of objects that doesn't fall in this category. 

\begin{definition}\label{def:filter_op}
	Let $\mathcal{F}$ be the $\textbf{filter}$ operator. It is given a set of polygons $S_P$ and returns a subset $S'_P$ of polygons:
	
	\begin{equation}
		\mathcal{F}(\cdot): S_P \rightarrow S'_P, S'_P \subseteq S_P
	\end{equation}
\end{definition}

Each instance of the workflow $\mathcal{W}_i$ except $\mathcal{W}_1$ is therefore associated a filter operator $\mathcal{F}_i$. The resulting algorithm is given in Algorithm \ref{algo:chaining_workflows} and has now reached an acceptable level of genericity.

\begin{algorithm} \label{algo:chaining_workflows}
	Construction of the $\mathcal{W}$ operator with $K$ instances of Algorithm \ref{algo:single_seg_several_classif}:

	\begin{enumerate}
		\item Execute the first workflow and save the results in the results set $R$: $R \leftarrow \mathcal{W}_1(\mathcal{I})$
		\item Create the polygons set and initializes it with the polygons found from the execution of $\mathcal{W}_1$: $S_P \leftarrow \left\{P_1, ..., P_N\right\}$
		\item For each $i \in \{2, ..., K\}$:
		\begin{enumerate}
			\item Extract polygons to be processed by $\mathcal{W}_i$: $S'_P \leftarrow \mathcal{F}_i(S_P)$
			\item For polygon $P \in S'_P$:
			\begin{enumerate}
				\item Execute workflow $\mathcal{W}_i$ on the image window and saves the results: $R \leftarrow R \cup \mathcal{W}_i(\mathcal{I}_P)$
				\item Add the extracted polygons to the polygons set: $S_P \leftarrow S_P \cup \left\{P_1, ..., P_M\right\}$
			\end{enumerate}
		\end{enumerate}
		\item Return the results set $R$
	\end{enumerate}
\end{algorithm}

\section{Implementation}
\label{sec:workflow_impl}
The workflow presented in Section \ref{sec:workflow_principle} is now defined. This section details the design choices and architecture of the implementation. In the subsequent sections, the term workflow will refer to the algorithm while the term framework will refer to the implementation.

\subsection{Requirements}
The main requirements for the framework are listed hereafter.

\paragraph{Genericity} As for the algorithm, the framework should be able to solve the widest possible range of objects detection and classification problems in any context. This property has more implication in the case of the framework design than for the algorithm design, especially when it comes to fixing the representation of the various involved data types (i.e. image, polygon,...).

\paragraph{Efficiency} While the framework has no control over the efficiency of the algorithms defined by the implementer (i.e. segmentation or classification procedures), the coordination of those algorithms should not induce a significant overhead in the overall execution. 

\paragraph{Large images} While large images handling was irrelevant at the algorithm design stage, it becomes critical at this point. To remain generic, the framework should not make any assumption about the size of the images to be processed. Especially, a whole image should not be assumed to fit into memory.

\paragraph{Robustness} The framework should be robust to errors. That is, a single error should not interrupt the whole execution. For instance, if the framework executes a set of independent computations and one of them fails, it should only be stopped if this failure is unrecoverable and affects all the other computations. Otherwise, the failure should be reported and those others computations should execute until completion. 

\paragraph{Transparency} The framework should provide a built-in way to communicate its progress, the duration of each steps as well as the errors it encounters with the user. The level of verbosity of this communication tool should be adjustable. 

\paragraph{Parallelism} Whenever possible the framework should take advantage of parallelism to reduce its execution time. However, the implementer should be given a way to disable parallelism and switch to sequential execution. Moreover, the implementer should be able to change the level of parallelism (i.e. the number of jobs on whi).

\paragraph{Ease of use} The work of the implementer should be kept as minimal as possible. He should only have to define the logic of problem dependent elements of the workflow: segmentation, dispatching rules, classifiers,... 

\subsection{Language}
\label{ssec:language}
The first design choice occurring in the implementation of an existing algorithm is obviously the language in which this implementation will be made. As far as the workflow is concerned, the chosen language was Python. Indeed, this language provides a very complete environment for developing objects detection and segmentation algorithms which would obviously contribute to the overall ease of use the framework. 

First of all, the language has many features which allows developers to quickly come up with solutions to problems. Especially, it is strongly and dynamically typed, multi-paradigm (imperative, functional, object oriented,...), interactive (it can be used in an interactive console), interpreted and garbage-collected. It also support usual data structures such as lists, arrays, dictionaries and sets natively and provide operations for manipulating them in a concise way. 

In addition to its built-in features, Python has become a great language for scientific computing as it has been augmented with excellent open source libraries over the years. First, the SciPy ecosystem which includes the SciPy \cite{oliphant:2007} and NumPy \cite{vanderwalt:2011} libraries. The first is a collection of numerical algorithms and domain-specific toolboxes (signal processing, optimization, statistics,...). The second is a fundamental package for numerical computations which provides an efficient representation of multi-dimensional arrays and operations on them. Built on top of the SciPy ecosystem comes Scikit-Learn \cite{pedregosa:2011}, a library that provides simple, efficient and reusable tools for data mining and machine learning. Image processing is not outdone with a Python binding for the huge OpenCV library \cite{opencv_library}. Two alternatives are scikit-image \cite{scikit-image} which is built on top of the SciPy ecosystem or the Pillow library \cite{pillow}. All of them provide a collection of well-known image processing algorithms. Another useful library is Shapely \cite{shapely} which provides a representation for geometric objects such as polygons as well as operations to apply on them. 

Then, Python was also chosen because the workflow was implemented to be integrated with Cytomine (see Section ??). Particularly, the final goal was the detection and classification of objects in images stored on Cytomine servers. As those images and their metadata are exposed through an API interfaced by a Python client, it was essential that the workflow could use this client to communicate with the back-end. As the Cytomine client was implemented in the version 2.7.11 of Python, this version was also used for developing the library. 

\subsection{Software architecture}
As one of the main goal of this implementation is to reuse it for solving other problems, the framework was implemented as a Python library and released on GitHub ?? url ??.

To ensure code modularity, classes were organized in packages. To prevent any name clashes with other Python libraries, those packages were placed in a root package called \texttt{sldc}. 

\subsubsection{Image representation} 
The image representation design is a critical point of the framework architecture. Indeed, on the one hand, it should be abstract enough so that implementers can apply the workflow on images in any format. On the other hand, it should provide access to a concrete representation available to the framework. Indeed, some steps need to access this representation to extract useful data. This is the case for location which processes the binary mask to extract polygons. 

The representation should also provide a way of extracting sub-windows from an image. The need for this feature is twofold. First, it is needed by the workflow (see Definition \ref{def:image_window}). Then, it could be used to address the large images handling requirement and to overcome the fact that a whole image is not assumed to fit into memory. The idea is to split an image into smaller chunks called tiles which could be loaded into memory and processed one after another. Especially, the tiles would be applied the first part of the workflow that is segmentation and location. As the polygons of each tile are extracted independently, it might occur that a single object of interest which spreads over several tiles ends up being splitted into several polygons. To make sure there is a one to one relationship between a polygon and an object, an additional step must be inserted before the dispatching. It would consist in merging the polygons representing a same object. This step is detailed in \ref{sssec:merging}.

The abstract image representation and related classes were implemented into the \texttt{sldc.image} package presented in the UML diagram shown in Figure \ref{fig:uml_image_package}.

The \texttt{Image} class is the abstract image representation mentioned above. It provides three abstract methods for checking image dimensions (width, height and number of channels) and a fourth one which is expected to return the concrete representation of the image as a Numpy multi-dimensional array. Using Numpy arrays was motivated by the fact that they are compatible with the various image processing libraries presented in Section \ref{ssec:language}. This structure already solves the first issue presented in this Section. Especially, the implementer image format should be encapsulated into an extension of \texttt{Image} which would implement the conversion from this format to the Numpy array representation.

An image window is materialized by the \texttt{ImageWindow} class which is based on the decorator pattern. It stores information about the position and size of the window as well as a reference to the parent image. Especially, location and size are respectively represented by coordinates of the first top left pixel included in the window (coordinates are referenced to the top left corner of the parent image) and by the window width and height. As an image window instance provides a level of indirection on top of another image, some methods are provided to fetch the base image as well as the absolute offset\footnote{The absolute offset is the offset of the window referenced to the base image's top left pixel. It is different from the image window offset if its parent image is also an image window.}.

A tile is also represented by a class named \texttt{Tile} which extends \texttt{ImageWindow} and augment it with an integer identifier field. As tiles can potentially be derived, a \texttt{TileBuilder} interface was developed. As suggested by the name, a class implementing this interface is responsible for building specific tile objects. This structure is actually an application of the factory method pattern which has the advantage of allowing the framework to build tile objects while remaining unaware about the class that will actually be instantiated.

Finally, to make it easier to iterate over the tiles of an image two classes were developed : \texttt{TileTopology} and \texttt{TileTopologyIterator}. The first is responsible for generating a set of tiles that fully covers an image. To generate this set of tiles, the topology object relies on three parameters : the tile maximum width and height and the overlap. The overlap parameter allows the merging procedure to be simpler as polygons corresponding to a same object which is spread over several tiles will have a geometric intersection. However, this parameter should be set carefully because it induces some additional computations as some parts of the image will be present in several tiles. The \texttt{TileTopology} also provides methods for checking whether a tile is the neighbour of another. The \texttt{TileTopologyIterator} allows to iterate over the tiles defined by a tile topology.

\begin{figure}[h]
	\center 
	\includegraphics[scale=0.9]{image/uml_image_package.png}
	\caption{Image representation classes - package \texttt{sldc.image}}
	\label{fig:uml_image_package}
\end{figure}

\subsubsection{Segmentation}
As explained in Section \ref{ssec:single_single}, the segmentation is not fixed by the framework and the implementer is expected to provide its own implementation. To represent this constraint in the framework, a \texttt{Segmenter} interface was defined. It provides a single method, \texttt{segment}, which receives a Numpy representation of the image and is expected to return another Numpy array storing the binary mask marking the objects of interest contained in this image. The binary mask however doesn't conform strictly to Definition \ref{def:segmentation_op} as pixels belonging to an object of interest are marked with the integer value 255 (which corresponds to white in the grayscale color space) instead of 1. 

\begin{figure}[h]
	\center 
	\includegraphics[scale=0.9]{image/uml_segmenter_package.png}
	\caption{Package \texttt{sldc.segmenter}}
	\label{fig:uml_segmenter_package}
\end{figure}

\subsubsection{Location}

\subsubsection{Merging}
\label{sssec:merging}

\subsubsection{Dispatch and classification}

\subsubsection{Workflow}

\subsubsection{Workflow chain}

\subsection{How to use the framework}
A toy example: finding disks in an image with grey background and guessing whether they're black or white 
	\newpage
	% End chapter 2 : generic workflow ========
	
	% Start chapter 3 : thyroid case ==========
	\chapter{\textit{SLDC} at work : the thyroid case}
	\section{Cytomine}
Presentation of cytomine 
\section{Implementation issues}
Presentation of implementation issues related to the thyroid case (image size, over HTTP, image quality, human annotation vs computer annotation, presence of inclusions in patterns, dispatching ...)
\section{Implementation}
Actual implementation of the processing using the workflow
\section{Performance analysis}
\subsection{Detection}
\subsection{Execution time}
	\newpage
	% End chapter 3 : thyroid case ============
	
	% Start conclusion ========================
	\chapter{Conclusion}
	This thesis proposes \textit{SLDC}, a generic framework for object detection and classification in mutli-gigapixel images. 
It provides implementers with a concise way of formulating their algorithm by declaring only problem dependent-components: segmentation procedures and classification models. Behind the scenes, the framework takes care of problem-independent concerns. For instance, in order to avoid loading the full image into memory, it splits this image in tiles which are processed independently. Parallelism is also encapsulated by the framework which applies it to accelerate tiles processing. Are also provided: a powerful and customizable logging system informing the user about errors and overall progress, a way of executing several workflows one after another on a same image and robustness so that errors of which the impact is negligible does not stop the whole program. The framework is available on GitHub as a Python library. 

The framework was then applied to a real-world problem, thyroid nodule malignancy diagnosis, in order to assess its performances. Especially, a workflow developed for this problem in a previous master thesis was analysed, improved and re-implemented using the framework. 

The results are promising: the effective execution time of the workflow was less than 10 minutes on a 8 gigapixels image (executed on 32 processes). This time is mostly due to design choice linked to the implementation and the framework only induces a negligible overhead. Some improvements can be done both at the framework and workflow levels. Especially, some other operations of the former could be parallelized and the current parallelization could be optimized.

As far as the thyroid case is concerned, the developed workflow does not provide a production-ready solution yet because it sometimes fails at detecting objects of interest and produces an important number of false positives. However, the analysis provided in this thesis already points out elements which needs to be improved (segmentation procedures, classification models,...) providing a baseline for any further development on this case.

	% End conclusion ==========================
	
	\appendix
	
	% Start tile topology appendix ============
	\chapter{Tile topology}
	\label{apdx:tile_topology}

As presented in Section \ref{sssec:work_image_repr}, the tile topology objects associate unique increasing identifiers to tiles. Using this representation allows to reach a $\mathcal{O}(1)$ time complexity for all the methods of the class \texttt{TileTopology}. Indeed, the results produced by those methods can be computed using simple formulas. In the following formulas, $i$ refers to a tile identifier:

\begin{itemize}
	\item The number $t_{row}$ of tiles on a row is given by:
	\begin{equation} \label{eqn:nb_tile_on_row}
		t_{row} = \left\{
		\begin{array}{cl}
			\left\lceil \dfrac{w - o_p}{w_m - o_p}\right\rceil&\text{, if } w > w_m\\
			1&\text{, otherwise}\\
		\end{array}
		\right.	
	\end{equation}
	\item The number $t_{col}$ of tiles on a column is given by Equation \ref{eqn:nb_tile_on_row} applied to the image height $h$ and maximum tile height $h_m$ instead of $w$ and $w_m$. 
	\item The total number $t$ of tiles in the tile topology is simply $t_{row} \times t_{col}$.
	\item The neighbour tiles identifiers can be obtained by performing subtractions and additions. For instance, for a tile  which is not on the edge of the image, the identifiers of its left, top, right and bottom neighbours are respectively $i - 1$, $i - t_{row}$, $i + 1$, $i + t_{row}$.
	\item The tile offset $(t_{\text{off},x}, t_{\text{off},y})$ can be retrieved as follows:
	\begin{eqnarray}
		t_{\text{off},x} = (t_{row} - o_p) \times \left[(i-1) \mod t_{row} \right] \\
		t_{\text{off},y} = (t_{col} - o_p) \times \left\lfloor\frac{i - 1}{t_{row}}\right\rfloor
	\end{eqnarray} 
\end{itemize}
	\newpage
	% End tile topology appendix ==============
	
    % Start ontology ==========================
	\chapter{Ontology}
	\label{app:ontology}
The ontology associated with the Thyroid project on Cytomine is the following:

\begin{enumerate}
	\item Architectural patterns:
	\begin{itemize}
		\item Normal follicular architectural pattern
		\item Proliferative follicular architectural pattern
		\item Proliferative follicular architectural pattern (minor sign)
	\end{itemize}
	\item Nuclear features:
	\begin{itemize}
		\item Papillary cell NOS
		\item Normal follicular cells
		\item Normal follicular cell with pseudo-inclusion (artefact)
		\item Papillary cell with ground glass nuclei
		\item Papillary cell with nuclear grooves
		\item Papillary cell with inclusion
	\end{itemize}
	\item Others:
	\begin{itemize}
		\item Macrophages
		\item Red blood cells
		\item PN (polynuclear)
		\item Colloid
		\item Artefacts
		\item Background
	\end{itemize}
\end{enumerate}

% TODO add images
	\newpage
	% End ontology ============================
	
	% Start Pyxit =============================
	\chapter{Random subwindows}
	\label{app:random_subwindows}
Random subwindows \cite{Maree201617} is an image classification algorithm. The first step of the algorithm consists in transforming the $N$ input images. This is done by extracting a set of $N_w$ random subwindows from each image. A random subwindow is a square patch of random size extracted at a random position in an image. The extracted windows are then resized to a fixed size patch $(w, h)$. Those transformation operations generates a dataset containing $N\times N_w$ objects and $w \times h$ attributes. 

The second step consists in passing this dataset to a classifier which will actually predict the image's classification label from those subwindows. In \cite{Maree201617}, two classification methods are proposed.

The first uses extremely randomized trees \cite{Geurts2006} as direct classifier: that is, each window is predicted a label and the full image label is determined by a majority vote over the predicted classes of this image's windows.

The second variant uses extremely randomized trees as feature learner rather than a direct classifier and relies on a SVM classifier to produce the prediction. 
In this variant each image is represented as a vector of which the dimensionality equals the number of terminal nodes in the ensemble of randomized trees and where the $i^{th}$ feature is the number of windows that reached the $i^{th}$ leaf node of the forest divided by the total number of windows. This vector is then passed to the SVM classifier to predict the image classification label.




	\newpage
	% End Pyxit ===============================
	
	% Start Cross validation ==================
	\chapter{Cross validation}
	\label{app:parameters}

This chapter presents the parameters that were tuned by the cross-validation procedure presented in Section \ref{sssec:thyroid_cv}. Each classifier was built using the two variants of the random subwindows algorithm , ET-FL and ET-DIC, and for each variant, the model was learned on two different learning sets: one with the reviewed annotations and one without. The complete lists of parameters are given in Figure \ref{tab:app_disp_classif_tuned} for the dispatching classifier, in Figure \ref{tab:app_cell_classif_tuned} for the cell classifier and in Figure \ref{tab:app_pattern_classif_tuned} for the pattern classifier.

\begin{table}
	\center 
	\begin{tabular}{|c|cccc|}
		\hline
		Parameters & ET-DIC & ET-DIC (r) & ET-FL & ET-FL (r) \\
		\hline		
		\texttt{window size} & \multicolumn{4}{|c|}{(0.3, 0.8), (0.3, 1.0), (0.5, 0.8), (0.5, 1.0)} \\
		\texttt{colorspace} & \multicolumn{4}{|c|}{HSV, normalized RGB} \\
		\multirow{2}{*}{\texttt{min\_sample\_split}} & $\{1, 91, 906,$ &  $\{1, 291, 2913$ & \multirow{2}{*}{91} & \multirow{2}{*}{291}\\
		& $1812, 4530\}$ & $5825, 14563 \}$ & & \\
		\texttt{max\_features} & \multicolumn{4}{|c|}{$\left\{1, 28, 384, 768\right\}$}\\ 
		\texttt{C} & / & / & \multicolumn{2}{c|}{$\left\{0.1, 1\right\}$}\\
		\hline
	\end{tabular}
	\caption{Dispatch classifier. Tuned parameters.}
	\label{tab:app_disp_classif_tuned}
\end{table}

\begin{table}
	\center 
	\begin{tabular}{|c|cccc|}
		\hline
		Parameters & ET-DIC & ET-DIC (r) & ET-FL & ET-FL (r) \\
		\hline		
		\texttt{window size} & \multicolumn{4}{|c|}{(0.6, 0.7), (0.6, 0.8)} \\
		\texttt{colorspace} & \multicolumn{4}{|c|}{HSV, normalized RGB} \\
		\multirow{2}{*}{\texttt{min\_sample\_split}} & $\{1, 108, 1075,$ &  $\{1, 156, 1564$ & \multirow{2}{*}{108} & \multirow{2}{*}{156}\\
		& $2150, 5375\}$ & $3127, 7818 \}$ & & \\
		\texttt{max\_features} & \multicolumn{4}{|c|}{$\left\{1, 28, 384, 768\right\}$}\\ 
		\texttt{C} & / & / & \multicolumn{2}{c|}{$\left\{0.1, 1\right\}$}\\
		\hline
	\end{tabular}
	\caption{Cell classifier. Tuned parameters.}
	\label{tab:app_cell_classif_tuned}
\end{table}


\begin{table}
	\center 
	\begin{tabular}{|c|cccc|}
		\hline
		Parameters & ET-DIC & ET-DIC (r) & ET-FL & ET-FL (r) \\
		\hline		
		\texttt{window size} & \multicolumn{4}{|c|}{(0.2, 0.4), (0.2, 0.3)} \\
		\texttt{colorspace} & \multicolumn{4}{|c|}{HSV, normalized RGB} \\
		\multirow{2}{*}{\texttt{min\_sample\_split}} & $\{1, 67, 675,$ &  $\{1, 76, 756$ & \multirow{2}{*}{91} & \multirow{2}{*}{291}\\
		& $1349, 3373\}$ & $1511, 3778\}$ & & \\
		\texttt{max\_features} & \multicolumn{4}{|c|}{$\left\{1, 28, 384, 768\right\}$}\\ 
		\texttt{C} & / & / & \multicolumn{2}{c|}{$\left\{0.1, 1\right\}$}\\
		\hline
	\end{tabular}
	\caption{Pattern classifier. Tuned parameters.}
	\label{tab:app_pattern_classif_tuned}
\end{table}


	\newpage
	% End Cross validation ====================
	
	% Start Performance table =================
	\chapter{Execution times}
	\label{app:exec_times}
In Tables \ref{tab:perf_time_size_and_jobs}, ..., are given detailed execution times for executions of the workflow on several images. All execution times are given in seconds. More details about the fields of Tables \ref{tab:perf_time_size_and_jobs} and \ref{tab:perf_time_typical} are given hereafter:

\begin{enumerate}
	\item \textbf{Run information}: global information about the run
	\begin{itemize}
		\item \textit{Run number}: a number associated with the execution in order to ease referencing those runs in the thesis
		\item \textit{Image width and height}: width and height (in pixels) of the image processed by the run
		\item \textit{Tile width and height}: width and height (in pixels) of the tiles used by the tile topology to break down the images in smaller chunks
		\item \textit{Tiles}: number of tiles containing in the topology
		\item \textit{Jobs}: number of processed assigned to the execution
		\item \textit{RAM}: maximum amount of memory available for the run to execute
	\end{itemize}
	\item \textbf{Polygons}: information about the polygons found by the run
	\begin{itemize}
		\item \textit{Found}: number of polygons found across all tiles 
		\item \textit{Merged}: number of polygons resulting from the merging phase
		\item \textit{Cell}: number of polygons dispatched to the cell classifier
		\item \textit{Pattern}: number of polygons dispatched to the pattern classifier
		\item \textit{Dispatched}: total number of dispatched polygons
	\end{itemize}
	\item \textbf{L-S-L}: execution times of the \textbf{L}oad-\textbf{S}egment-\textbf{L}ocate phase. This phase is parallelized.
	\begin{itemize}
		\item \textit{Loading}: total amount of time for loading tiles into memory (on separate processes)
		\item \textit{Segment}: total amount of time for segmenting the tiles (on separate processes)
		\item \textit{Location}: total amount of time for locating polygons in segmented tiles (on separate processes)
		\item \textit{Overall}: actual amount of time for processing all the tiles (wall-clock time)
	\end{itemize}
	\item \textbf{Dispatching}: execution times of the dispatching phase
	\begin{itemize}
		\item \textit{Cell model}: variant of the random subwindows algorithm used for dispatching polygons
		\item \textit{Fetch 1}: times needed for fetching the crops of the polygons to dispatch from the Cytomine server
		\item \textit{Cells}: amount of time needed for finding whether the polygons should be dispatched to the cell classifier or not
		\item \textit{Fetch 2}: time needed for fetching the crops of the polygons to dispatch to the pattern classifier. Normally, it should always be small as all the crops have already been downloaded and cached by the \textit{Fetch 1} step
		\item \textit{Patterns}: amount of time for finding whether the polygons should be dispatched to the pattern classifier or not
		\item \textit{Overall}: total amount of time for dispatching the polygons 
	\end{itemize}
	\item \textbf{Classification}: execution times of the classification phase
	\begin{itemize}
		\item \textit{Fetch 3}: times needed for fetching the crops of the polygons to be processed by the cell classifier. As for \textit{Fetch 2}, those times should be low. 
		\item \textit{Cells}: amount of time for classifying the cells 
		\item \textit{Fetch 4}: times needed for fetching the crops of the polygons to be processed by the cell classifier. As for \textit{Fetch 2}, those times should be low.
		\item \textit{Patterns}: amount of time for classifying patterns
		\item \textit{Overall}: total amount of time for classifying the dispatched polygons
	\end{itemize}
	\item \textbf{Net.}: for \textit{Network}, time spent for sending network requests and waiting for responses
	\begin{itemize}
		\item \textit{Caching}: amount of time needed for fetching and caching the tiles of the topology
		\item \textit{Upload}: amount of time needed for uploading the dispatched polygons to the Cytomine server
	\end{itemize}
	\item \textbf{Total}: 
	\begin{itemize}
		\item \textit{Not net. 1}: total execution time of the run from which was deduced the \textit{Caching}, \textit{Upload}, \textit{Fetch 1}, \textit{Fetch 2}, \textit{Fetch 3} and \textit{Fetch 4} execution times
		\item \textit{Not net. 2}: total execution time of the run from which was deduced the \textit{Caching} and \textit{Upload} execution times. 
		\item \textit{Overall}: total execution time of the run. Might not equal the sum of the various steps executions times. Indeed, some operations performed between those steps are not included in the corresponding execution times
	\end{itemize}		
	
\end{enumerate}

\begin{table}
	\footnotesize
	\center
	\begin{tabular}{|c|c|c|c|c|c|c|c|}
		\hline
		\parbox[t]{2mm}{\multirow{9}{*}{\rotatebox[origin=c]{90}{Run information}}} & Run nb. & 1 & 2 & 3 & 4 & 5 & 6 \\
		& Image & 728725 & 728725 & 728725 & 728725 & 728725 & 728725 \\
		& Width & 131072 & 131072 & 131072 & 131072 & 131072 & 131072 \\
		& Height & 57856 & 57856 & 57856 & 57856 & 57856 & 57856 \\
		& Tile width & 512 & 512 & 512 & 1024 & 1024 & 1024 \\
		& Tile height & 512 & 512 & 512 & 1024 & 1024 & 1024 \\
		& Tiles & 29900 & 29900 & 29900 & 7353 & 7353 & 7353 \\
		& Jobs & 16 & 32 & 64 & 16 & 32 & 64 \\
		& RAM (Go) & 22,37 & 50,77 & 72,48 & 21,56 & 41,74 & 73,60 \\
		&  &  &  &  &  &  & \\
		\parbox[t]{2mm}{\multirow{5}{*}{\rotatebox[origin=c]{90}{Polygons}}} & Found & 10009 & 10009 & 10009 & 8418 & 8418 & 8418 \\
		& Merged & 7294 & 7294 & 7294 & 7195 & 7195 & 7195 \\
		& Cell & 5172 & 5169 & 5169 & 5141 & 5118 & 5128 \\
		& Pattern & 1581 & 1567 & 1572 & 1528 & 1540 & 1554 \\
		& Dispatched & 6753 & 6736 & 6741 & 6669 & 6658 & 6682 \\
		&  &  &  &  &  &  & \\
		\parbox[t]{2mm}{\multirow{4}{*}{\rotatebox[origin=c]{90}{LSL}}} & Loading & 593.145 & 1544.394 & 773.857 & 537.775 & 951.596 & 1210.165 \\
		& Segment & 4801.758 & 6736.784 & 7321.667 & 5148.477 & 7376.851 & 7757.463 \\
		& Location & 2083.925 & 2492.705 & 2472.753 & 2033.105 & 2427.312 & 2690.361 \\		
		& \textbf{Overall} & \textbf{476.952} & \textbf{348.405} & \textbf{199.245} & \textbf{536.385} & \textbf{351.635} & \textbf{206.755} \\
		&  &  &  &  &  &  & \\
		& Merging & 14.324 & 14.451 & 18.086 & 40.640 & 40.605 & 40.774 \\
		&  &  &  &  &  &  & \\
		\parbox[t]{2mm}{\multirow{6}{*}{\rotatebox[origin=c]{90}{Dispatching}}} & Model & ET-FL & ET-FL & ET-FL & ET-FL & ET-FL & ET-FL \\
		& Fetch 1 & 251.098 & 5.148 & 1.678 & 108.064 & 5.766 & 1.485 \\
		& Cells & 765.858 & 758.031 & 739.323 & 762.242 & 758.993 & 741.434 \\
		& Fetch 2 & 0.750 & 1.079 & 0.959 & 1.077 & 0.860 & 0.830 \\
		& Patterns & 112.861 & 140.930 & 135.782 & 142.667 & 141.502 & 137.354 \\
		& \textbf{Overall} & \textbf{1130.706} & \textbf{905.329} & \textbf{877.897} & \textbf{1014.201} & \textbf{907.267} & \textbf{881.252} \\
		&  &  &  &  &  &  & \\
		\parbox[t]{2mm}{\multirow{7}{*}{\rotatebox[origin=c]{90}{Classification}}} & Model & ET-DIC & ET-DIC & ET-DIC & ET-DIC & ET-DIC & ET-DIC \\
		& Fetch 3 & 1.372 & 1.602 & 1.664 & 1.200 & 1.431 & 1.167 \\
		& Cells & 19.248 & 14.213 & 12.614 & 20.849 & 14.141 & 12.165 \\
		& Model & ET-DIC & ET-DIC & ET-DIC & ET-DIC & ET-DIC & ET-DIC \\
		& Fetch 4 & 0.667 & 0.729 & 0.851 & 0.582 & 0.781 & 0.623 \\
		& Patterns & 7.527 & 5.639 & 4.978 & 7.465 & 5.212 & 5.136 \\
		& \textbf{Overall} & \textbf{28.865} & \textbf{22.240} & \textbf{20.162} & \textbf{30.149} & \textbf{21.622} & \textbf{19.149} \\
		&  &  &  &  &  &  & \\
		\parbox[t]{2mm}{\multirow{2}{*}{\rotatebox[origin=c]{90}{Net.}}} & Caching & 6193.270 & 27.953 & 4.178 & 3858.679 & 10.786 & 4.255 \\
		& Upload & 4993.832 & 444.564 & 444.000 & 4039.734 & 471.113 & 435.173 \\
		&  &  &  &  &  &  & \\
		\parbox[t]{2mm}{\multirow{3}{*}{\rotatebox[origin=c]{90}{Total}}} & Not net. 1 & 1286,466 & 1143,162 & 976,950 & 1369,459 & 1172,111 & 1007,756 \\
		& Not net. 2 & 1652.464 & 1291.571 & 1116.922 & 1621.971 & 1321.591 & 1148.385 \\
		& \textbf{Overall} & \textbf{12839.566} & \textbf{1764.088} & \textbf{1565.100} & \textbf{9520.385} & \textbf{1803.489} & \textbf{1587.814} \\
		\hline
	\end{tabular}
	\caption{Effects of varying the tile sizes and the available number processes on the execution times. Test image (728725) has dimensions $57856 \times 131072$.}
	\label{tab:perf_time_size_and_jobs}
\end{table}

\begin{table}
	\footnotesize
	\center 
	\begin{tabular}{|c|c|c|c|c|c|c|c|}
	\hline
	 \parbox[t]{2mm}{\multirow{9}{*}{\rotatebox[origin=c]{90}{Run information}}} & Run nb. & 7 & 8 & 9 \\
	 & Image & 728744 & 716528 & 728725 \\
	 & Width & 172032 & 163840 & 131072 \\
	 & Height & 104704 & 95744 & 57856 \\
	 & Tile width & 1024 & 1024 & 1024 \\
	 & Tile height & 1024 & 1024 & 1024 \\
	 & Tiles & 17510 & 15390 & 7353 \\
	 & Jobs & 32 & 32 & 32 \\
	 & RAM (Go) & 79.45 & 73.13 & 66.38 \\
	 &  &  &  &  \\
	 \parbox[t]{2mm}{\multirow{5}{*}{\rotatebox[origin=c]{90}{Polygons}}} & Found & 57266 & 45617 & 8418 \\
	 & Merged & 54098 & 42294 & 7195 \\
	 & Cell & 42782 & 33042 & 5703 \\
	 & Pattern & 6274 & 6351 & 1079 \\
	 & Dispatched & 49056 & 39393 & 6782 \\
	 &  &  &  &  \\
	 \parbox[t]{2mm}{\multirow{4}{*}{\rotatebox[origin=c]{90}{LSL}}} & Loading & 1521.482 & 1395.434 & 968.262 \\
	 & Segment & 16492.846 & 13601.847 & 6378.335 \\
	 & Location & 5582.815 & 4963.792 & 2406.242 \\
	 & \textbf{Overall} & \textbf{757.256} & \textbf{670.970} & \textbf{351.508} \\
	 &  &  &  &  \\
	 & Merging & 102.940 & 183.515 & 40.991 \\
	 &  &  &  &  \\
	 \parbox[t]{2mm}{\multirow{6}{*}{\rotatebox[origin=c]{90}{Dispatching}}} & Model & ET-DIC & ET-DIC & ET-DIC \\
	 & Fetch 1 & 2498.633 & 1189.211 & 12.092 \\
	 & Cells & 139.267 & 119.410 & 25.621 \\
	 & Fetch 2 & 1.813 & 1.434 & 0.570 \\
	 & Patterns & 21.725 & 21.999 & 6.882 \\
	 & \textbf{Overall} & \textbf{881.252} & \textbf{1332.723} & \textbf{45.320} \\
	 &  &  &  &  \\
	 \parbox[t]{2mm}{\multirow{7}{*}{\rotatebox[origin=c]{90}{Classification}}} & Model & ET-DIC & ET-DIC & ET-DIC \\
	 & Fetch 3 & 9.052 & 5.620 & 1.389 \\
	 & Cells & 104.618 & 86.905 & 15.706 \\
	 & Model & ET-DIC & ET-DIC & ET-DIC \\
	 & Fetch 4 & 1.804 & 1.352 & 0.469 \\
	 & Patterns & 16.929 & 19.138 & 6.058 \\
	 & \textbf{Overall} & \textbf{132.793} & \textbf{113.265} & \textbf{23.678} \\
	 &  &  &  &  \\
	 \parbox[t]{2mm}{\multirow{2}{*}{\rotatebox[origin=c]{90}{Net.}}} & Caching & 29349.127 & 33666.674 & 16.729 \\
	 & Uploading & 3105.667 & 16592.120 & 450.024 \\
	 &  &  &  &  \\
	 \parbox[t]{2mm}{\multirow{3}{*}{\rotatebox[origin=c]{90}{Total}}} & Not net. 1 & 1145.342 & 1103.856 & 447.475 \\
	 & Not net. 2 & 3656.644 & 2301.474 & 461.995 \\
	 & \textbf{Overall} & \textbf{36111.438} & \textbf{52560.267} & \textbf{928.747} \\
	\hline
	\end{tabular}
	\caption{Typical executions of the workflow (slide processing only). Execution times are given in seconds.}
	\label{tab:perf_time_typical}
\end{table}


More details about the fields of Table \ref{tab:workflow_exec_pattern_incl} are given hereafter

\begin{enumerate}
	\item \textbf{Run information}: same as for Tables \ref{tab:perf_time_size_and_jobs} and \ref{tab:perf_time_typical}
	\item \textbf{Slide proc.}: information about execution of the first phase, i.e. slide processing
	\begin{itemize}
		\item \textit{Poly.}: number of polygons found and dispatched by the phase
		\item \textit{LSL}: total execution time for the \textit{Load/Segment/Locate} phase
		\item \textit{Merging}: total execution time for the merging phase
		\item \textit{Dispatch}: total execution time for the dispatch phase
		\item \textit{Classify}: total execution time for the classify phase
		\item \textit{Total}: total duration of the slide processing phase 
	\end{itemize}
	\item \textbf{Pattern proc.}: information about execution of the second phase, i.e. patterns processing
	\begin{itemize}
		\item \textit{Patterns}: number of patterns found by the previous phase and processed by this one
		\item \textit{Poly.}: number of polygons (i.e. cells) found and dispatched by the phase
		\item \textit{LSL}: total execution time for all the executed \textit{LSL} phases. In addition is given the average \textit{LSL} execution time and standard deviation for one pattern. 
		\item \textit{Merging}: total execution time for all the executed merging phases (+ average time and standard deviation for one pattern)
		\item \textit{Dispatch}: total execution time for all the executed dispatching phases (+ average time and standard deviation for one pattern)
		\item \textit{Classify}: total execution time  for all the executed classification phases (+ average time and standard deviation for one pattern)
		\item \textit{Total}: total execution time for the phase (+ average time and standard deviation for one pattern)
	\end{itemize}	 
	\item \textbf{Net.}: same as for Tables \ref{tab:perf_time_size_and_jobs} and \ref{tab:perf_time_typical}. \textit{Upload 1} refers to the upload of results from the slide processing and \textit{Upload 2} to that of the patterns processing.
	\item \textbf{Total}: same as for Tables \ref{tab:perf_time_size_and_jobs} and \ref{tab:perf_time_typical}. Note that in this case, \textit{No net. 1} also includes network time for fetching crops of the cells detected by the patterns processing.
\end{enumerate}
\begin{table}
	\footnotesize
	\center
	\begin{tabular}{|c|c|cc|cc|}
	\hline
	 \parbox[t]{2mm}{\multirow{9}{*}{\rotatebox[origin=c]{90}{Run information}}} & Run nb. & \multicolumn{2}{c|}{10} & \multicolumn{2}{|c|}{11} \\
	 & Image & \multicolumn{2}{c|}{716528} & \multicolumn{2}{|c|}{728725} \\
	 & Width & \multicolumn{2}{c|}{163840} & \multicolumn{2}{|c|}{131072} \\
	 & Height & \multicolumn{2}{c|}{95744} & \multicolumn{2}{|c|}{57856} \\
	 & Tile width & \multicolumn{2}{c|}{1024} & \multicolumn{2}{|c|}{1024} \\
	 & Tile height & \multicolumn{2}{c|}{1024} & \multicolumn{2}{|c|}{1024} \\
	 & Tiles & \multicolumn{2}{c|}{15390} & \multicolumn{2}{|c|}{7353} \\
	 & Jobs & \multicolumn{2}{c|}{50} & \multicolumn{2}{|c|}{64} \\
	 & RAM (Go) & \multicolumn{2}{c|}{138.66} & \multicolumn{2}{|c|}{178.47} \\
	 &  &  &  &  & \\
	 \parbox[t]{2mm}{\multirow{6}{*}{\rotatebox[origin=c]{90}{Slide proc.}}} & Poly. & \multicolumn{2}{c|}{39430} & \multicolumn{2}{|c|}{6777} \\
	 & LSL & \multicolumn{2}{c|}{732.967} & \multicolumn{2}{|c|}{271.944} \\
	 & Merging & \multicolumn{2}{c|}{208.819} & \multicolumn{2}{|c|}{47.122} \\
	 & Dispatch & \multicolumn{2}{c|}{193.681} & \multicolumn{2}{|c|}{41.614} \\
	 & Classify & \multicolumn{2}{c|}{134.661} & \multicolumn{2}{|c|}{40.890} \\
	 & \textbf{Total} & \multicolumn{2}{c|}{\textbf{1271.042}} & \multicolumn{2}{|c|}{\textbf{404.742}} \\
	 &  &  &  &  & \\
	 \parbox[t]{2mm}{\multirow{7}{*}{\rotatebox[origin=c]{90}{Pattern proc.}}} & Patterns & \multicolumn{2}{c|}{6323} & \multicolumn{2}{|c|}{1080} \\
	 & Poly. & \multicolumn{2}{c|}{39633} & \multicolumn{2}{|c|}{13269} \\
	 & LSL & 9500.147 & {\tiny1.50 $\pm$ 0.34} & 2420.109 & {\tiny2.241 $\pm$ 1.135} \\
	 & Merging & 225.594 & {\tiny0.035 $\pm$1.04} & 113.494 & {\tiny0.105 $\pm$ 0.934} \\
	 & Dispatch & 7.734 & {\tiny0.00124 $\pm$ 0.00267} & 1.196 &  {\tiny0.00122 $\pm$ 0.00191}\\
	 & Classify & 7523.256 & {\tiny1.44 $\pm$ 1.2316} & 1757.388 & {\tiny2.180 $\pm$ 2.443} \\
	 & \textbf{Total} & \textbf{17319.615} & {\tiny\textbf{2.678 $\pm$ 1.305}} & \textbf{4307.111} & {\tiny\textbf{3.934 $\pm$ 3.293}} \\
	 &  &  &  &  & \\
	 \parbox[t]{2mm}{\multirow{3}{*}{\rotatebox[origin=c]{90}{Net.}}} & Caching & \multicolumn{2}{c|}{47.876} & \multicolumn{2}{|c|}{3.358}\\
	 & Upload 1 & \multicolumn{2}{c|}{20399.008} & \multicolumn{2}{|c|}{450.024} \\
	 & Upload 2 & \multicolumn{2}{c|}{2762.877} & \multicolumn{2}{|c|}{1138.706} \\
	 &  &  &  &  & \\
	 \parbox[t]{2mm}{\multirow{2}{*}{\rotatebox[origin=c]{90}{Tot.}}} & No net. 1 & \multicolumn{2}{c|}{18551.736} & \multicolumn{2}{|c|}{4703.322} \\
	 & \textbf{Overall} & \multicolumn{2}{c|}{\textbf{41761.498}} & \multicolumn{2}{|c|}{\textbf{6295.409}} \\
	\hline
	\end{tabular}
	\caption{Typical executions of the workflow (including pattern processing). Execution times are given in seconds.}
	\label{tab:workflow_exec_pattern_incl}
\end{table}
	\newpage
	% End Performance table ===================
	
	% Start list of figures ===================
	%\chapter*{Notations}
	%\begin{tabular}{lcl}
	\multicolumn{3}{l}{\textbf{Image :}} \\ 
	& $\mathcal{I}$ & An image \\
	& $w$ & The width of an image \\
	& $h$ & The height of an image \\
	& $c$ & The number of channels of an image \\
	& $\mathcal{I}_{hw}$ & An two dimensional image of width $w$ and height $h$ \\	
	& $p_{ij}$ & A pixel at row $i$ and column $j$ of a two dimensional image \\
	& $\mathcal{B}$ & A binary image \\
	& $b_{ij} \in \{0, 1\}$ & A pixel of a binary image\\
	& $P$ & A polygon \\
	\multicolumn{3}{l}{\textbf{Machine learning :}} \\ 
	& $T(\cdot)$ & A classifier \\
	& $C$ & A classification label \\
\end{tabular}
	%\newpage
	% End list of figures =====================


	% Start list of tables ====================
	\listoftables
	% End list of tables ======================
	
	% Start list of figures ===================
	\listoffigures
	% End list of figures =====================
	
	% Start bibliography ======================
	\printbibliography[heading=bibintoc]
	% End bibliography ========================
\end{document}
